\documentclass[12pt]{article}

\usepackage{answers}
\usepackage{setspace}
\usepackage{graphicx}
\usepackage{enumitem}
\usepackage{multicol}
\usepackage{mathrsfs}
\usepackage[margin=1in]{geometry} 
\usepackage{amsmath,amsthm,amssymb}
\usepackage{circuitikz} % PACKAGE for circuit desing
 % http://texdoc.net/texmf-dist/doc/latex/circuitikz/circuitikzmanual.pdf
\newcommand{\N}{\mathbb{N}}
\newcommand{\Z}{\mathbb{Z}}
\newcommand{\C}{\mathbb{C}}
\newcommand{\R}{\mathbb{R}}

\DeclareMathOperator{\sech}{sech}
\DeclareMathOperator{\csch}{csch}
 
\newenvironment{theorem}[2][Theorem]{\begin{trivlist}
\item[\hskip \labelsep {\bfseries #1}\hskip \labelsep {\bfseries #2.}]}{\end{trivlist}}
\newenvironment{definition}[2][Definition]{\begin{trivlist}
\item[\hskip \labelsep {\bfseries #1}\hskip \labelsep {\bfseries #2.}]}{\end{trivlist}}
\newenvironment{proposition}[2][Proposition]{\begin{trivlist}
\item[\hskip \labelsep {\bfseries #1}\hskip \labelsep {\bfseries #2.}]}{\end{trivlist}}
\newenvironment{lemma}[2][Lemma]{\begin{trivlist}
\item[\hskip \labelsep {\bfseries #1}\hskip \labelsep {\bfseries #2.}]}{\end{trivlist}}
\newenvironment{exercise}[2][Exercise]{\begin{trivlist}
\item[\hskip \labelsep {\bfseries #1}\hskip \labelsep {\bfseries #2.}]}{\end{trivlist}}
\newenvironment{solution}[2][Solution]{\begin{trivlist}
\item[\hskip \labelsep {\bfseries #1}]}{\end{trivlist}}
\newenvironment{problem}[2][Problem]{\begin{trivlist}
\item[\hskip \labelsep {\bfseries #1}\hskip \labelsep {\bfseries #2.}]}{\end{trivlist}}
\newenvironment{question}[2][Question]{\begin{trivlist}
\item[\hskip \labelsep {\bfseries #1}\hskip \labelsep {\bfseries #2.}]}{\end{trivlist}}
\newenvironment{corollary}[2][Corollary]{\begin{trivlist}
\item[\hskip \labelsep {\bfseries #1}\hskip \labelsep {\bfseries #2.}]}{\end{trivlist}}
 
\begin{document}
 
% --------------------------------------------------------------
%                         Start here
% --------------------------------------------------------------
 
\title{Electronic\\
Formulas and utilities}
\author{Tommaso Polloni\\
and } %inserisci il tuo nome Ale 
 
 The very first idea is to make this document a learning tool for git and Electronics. I will be very proud of ourselves if we
 reach to maintain English as its primary language. Let's do it!
\maketitle


\section{Semiconductors}
Semiconductors are material with an electrical conductivity value falling between that of a conductor and an insulator.
An important property of semiconductors is that, during a temperature increment, their resistance decreases.
Common semiconductors material are:
\begin{itemize}
	\item Silicon, Si. Atomic number = 14. IV ($14^{th}$) group.
	\item Germanium, Ge. Atomic number = 32. IV ($14^{th}$) group
	\item Gallium Arsenide, GaAs, a compound of Gallium and Arsenic.
\end{itemize}

\subsection{Doping}
The doping process consists in significantly change the number of \textit{electrons }or \textit{holes} in a semiconductor.
This result is achieved inserting atoms of the doping element in the crystalline arrangement of the doped.
\subsubsection{N-type doping}
Atoms with 5 valence electrones have an electron more than the silicon atoms. Pentavalent atoms share an electron with each of the four neighbouring silicon atoms, while the extra electron is free to move and it's responsible for conduction.
An example of a 5th gruop element is phosphorous.
\subsubsection{Utilities and numbers}
\begin{itemize}
	\item $N_{d} \approxeq 10^{12} : 10^{20} cm^{-3}$ number of donors
	\item $p 	[\dfrac{electrons}{cm^3}]$ number of holes 
	\item $n 	[\dfrac{electrons}{cm^3}]$ number of electrons 
\end{itemize}

TO ADD: the part regarding mobility and resistance.

\section{Diodes}
Diodes are defined as a p-n junction.
\begin{circuitikz}2
	\draw (0,0) to[Do, l=$Diode$, i>^=$i$, v = $v$] (2,0);3
\end{circuitikz}
Is is made by two parts: the left part (according to the figure) is a p-doped semiconductor in which there are holes left.
The other part, on the contrary, is n-doped in which there are electrons left. 

% --------------------------------------------------------------
%                         Sample structures
% --------------------------------------------------------------


\newpage
%Below is an example of the problem environment
\begin{problem}{6}
\begin{enumerate}[label=\alph*)]
    \item Suppose an entire function $f$ is bounded by $M$ along $|z|=R$. Show that the coefficients $C_k$ in its power series expansion about $0$ satisfy
    \[
    |C_k|\leq\frac{M}{R^k}.
    \]
    \item Suppose a polynomial is bounded by $1$ in the unit disc. Show that all its coefficients are bounded by 1.
\end{enumerate}
\end{problem}

%Below is the solution environment
\begin{solution}{}
Part a): Since $f$ is an entire function it can be expressed as an infinite power series, i.e.
\[
f(z)=\sum_{k=0}^\infty\frac{f^{(k)}(0)}{k!}z^k=\sum_{k=0}^\infty C_kz^k.
\]
If we recall Cauchy's Integral we have
\[
f(z)=\frac{1}{2\pi i}\int_\gamma\frac{f(w)}{w-z}\ dw,
\]
carefully notice that $\frac{1}{w-z}=\frac{1}{w}\cdot\frac{1}{1-\frac{z}{w}}$ can be written as a geometric series. We have

%The align environment with no label
\begin{align*}
\frac{1}{2\pi i}\int_\gamma\frac{f(w)}{w-z}\ dw &=\frac{1}{2\pi i}\int_\gamma\left\lbrace\frac{f(w)}{w}\cdot\left(\frac{1}{1-\frac{z}{w}}\right) \right\rbrace\ dw\\[8pt]
&=\frac{1}{2\pi i}\int_\gamma\left\lbrace\frac{f(w)}{w}\cdot\left(1+\frac{z}{w}+\frac{z^2}{w^2}+\frac{z^3}{w^3}+\cdots\right) \right\rbrace\ dw\\[8pt]
&=\left(\frac{1}{2\pi i}\int_\gamma \frac{f(w)}{w}\ dw\right)z^0+\left(\frac{1}{2\pi i}\int_\gamma \frac{f(w)}{w^2}\ dw\right)z^1+\left(\frac{1}{2\pi i}\int_\gamma \frac{f(w)}{w^3}\ dw\right)z^2\cdots
\end{align*}
Now take the modulus of $C_k$ to get
\[
|C_k|=\left\lvert\frac{1}{2\pi i}\int_\gamma \frac{f(w)}{w^{k+1}}\ dw \right\rvert\leq\frac{1}{2\pi}\int_\gamma\frac{|f(w)|}{|w^{k+1}|}\ |dw|\leq \frac{M}{2\pi}\int_\gamma\frac{|dw|}{|w^{k+1}|}
\]
Then integrate along $\gamma(\theta)=Re^{i\theta}$ for $\theta\in [0,2\pi]$ to get
\[
|C_k|\leq \frac{M}{2\pi}\int_0^{2\pi}\frac{|iRe^{i\theta}\ d\theta|}{|R^{k+1}e^{ik\theta}|}=\frac{M}{2\pi\cdot R^k}\int_0^{2\pi}\ d\theta=\frac{M}{R^k}.
\]
Hence, $|C_k|\leq \frac{M}{R^k}$.
\end{solution}
\pagebreak

\end{document}
